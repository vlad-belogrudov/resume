%%%%%%%%%%%%%%%%%%%%%%%%%%%%%%%%%%%%%%%%%%%%%%%%%%%%%%%%%%
%                                                        %
%     resume of Vladislav Belogrudov, May of 2017        %
%                                                        %
%%%%%%%%%%%%%%%%%%%%%%%%%%%%%%%%%%%%%%%%%%%%%%%%%%%%%%%%%%

\documentclass[a4paper,12pt,]{article}

\usepackage[dvips]{graphics}
\usepackage[normalem]{ulem}
\usepackage[colorlinks=true,urlcolor=blue]{hyperref}

\setlength{\topmargin}{-10mm}
\setlength{\headsep}{15mm}
\setlength{\topskip}{0mm}
\setlength{\oddsidemargin}{0mm}
\setlength{\evensidemargin}{0mm}
\setlength{\parindent}{0mm}
\setlength{\textwidth}{170mm}
\setlength{\textheight}{225mm}
\setlength{\footskip}{12mm}

\usepackage{fancyhdr}
\pagestyle{fancy} 
\lhead{}
\chead{e-mail: vlad.belogrudov@gmail.com \hfill\ mobile: +7 911 991 0894} 
\rhead{}
\lfoot{}
\cfoot{\thepage}
\rfoot{}

\begin{document}
  \parbox{\textwidth} {
    \parbox[b]{145mm}{
      { \bfseries \LARGE Vladislav Belogrudov}
      \vspace{5ex}

      \large \em
      Software development professional with a diverse project experience and a passion for innovation and new technologies
      \vspace{6ex}
    }
    \hfill
  }

% \renewcommand{\labelitemi}{$\star$}
  \uline{ \bfseries{SUMMARY OF QUALIFICATIONS} }
  
  \begin{itemize}
    \item Deep expertise in cloud computing and big data processing.

    \item Solid knowledge of storage systems, networking and virtualization.

    \item 20 years working closely with Linux/UNIX systems,
      designing, developing, testing and maintaining software systems.
     
    \item Variety of roles - manager, developer, QA engineer, lecturer.

    \item Excellent communicational skills, can take responsibility for projects and teams.
      
    \item Creative, eager to learn new things, to do challenging tasks, goal-oriented.

    \item Python, C/C++, Go, R, Java, Ansible, Docker, Kubernetes, variety of other languages and tools.
  \end{itemize}
  
  \vspace{1ex}

  \uline{ \bfseries{PROFESSIONAL EXPERIENCE} }

  \begin{description}

  \item{\bfseries Oracle Development SPb, Russia} \hfill September 8, 2014 - current \\
    {\em Principal Software Developer, Software Development Manager}

    \begin{itemize}

    \item Leaded development of the first commercially available OpenStack implementation completely packaged as Docker instances.

    \item Designed and implemented Oracle OpenStack features:
        \begin{itemize}
        \item containerized OpenStack services
        \item automated deployment via Ansible
        \item implemented online update, high availability and scalability
        \end{itemize}

    \item Integrated MySQL Cluster into Oracle OpenStack distribution to provide A/A database backend.

    \item Managed Oracle OpenStack QA team:
        \begin{itemize}
        \item created test plans, took responsibility for the release quality
        \item provided critical bugfixes for Oracle OpenStack and upstream project
        \item facilitated communication between product management, development, QA and release support
		\end{itemize}

    \item Since 2018, managed QA automation teams:
        \begin{itemize}
        \item Oracle Cloud (development testing)
        \item Oracle Linux distribution testing
        \item Kubernetes testing and test automation, CI/CD
        \end{itemize}

    \item Actively contributed to \href{https://docs.openstack.org/developer/kolla-ansible}{OpenStack Kolla project}: design ideas, feature implementation, code review and bug fixes:\\
       \url{http://stackalytics.com/?release=all&user_id=vlad-belogrudov}

    \item OpenStack Summits talks:
       \begin{itemize}
       \item \href{https://youtu.be/RJf7cwkytOE}{High Availability with MySQL Active/Active Clustering}
       \item \href{https://youtu.be/wzN3RHnVWdQ}{Containers - Silver Bullet for OpenStack Deployment?}
       \end{itemize}
    \end{itemize}

  \item{\bfseries Cork Institute of Technology, Ireland} \hfill September 2, 2013 - September 1, 2014 \\
    {\em Senior Research Fellow, Lecturer}

    \begin{itemize}

    \item Developed cloud-based software solution for next generation diagnostics in infectious diseases - \url{www.cloudxi.eu}. Designed and implemented highly scalable distributed and parallel processing bioinformatic systems (capable to process extremely large genome sequences). Implemented load-balancing of bioinformatic workflows among servers, designed management layer to abstract bioinformatic tools execution. Intensively used XenServer, LXC, Python web services.

    \item Evaluated Hadoop, VMware ESXi, Hyper-V, XenServer and OpenStack to build bioinformatic services.

    \item Designed and developed Hadoop based solution to parallelize NCBI BLAST to make search in extremely large nucleotide and protein databases.

    \item Installed and configured hardware and software systems for research laboratory - servers, networking, many bioinformatic tools.

    \item Developed and taught \href{https://courses.cit.ie/index.cfm/page/module/moduleId/13651}{Distributed Data Management} course to Data Analytics students. Main topics covered in the course are Big Data, NoSQL databases, Hadoop, Python and R streaming in Hadoop.

    \item Supervised student projects (Big Data and Hadoop).

    \item Supervised networking labs (Cisco).

    \end{itemize}

  \item{\bfseries EMC, Saint Petersburg, Russia} \hfill  May 26, 2008 - August 16, 2013 \\
    {\em Principal Software Engineer}  

    \begin{itemize}

    \item Designed and developed automated test framework for CSX (Common Execution Environment - technology and components for creation of platform independent Data Path software). 

    \item Developed Data Path components for EMC VNXe storage systems.

    \item Prepared and conducted seminars on storage technologies, virtualization and cloud computing to colleagues and students at Saint Petersburg universities. 

    \item Mentored student projects (storage and retrieval of images in Atmos cloud, approximate search engine, approximate file system deduplicator, virtual cloud building, cloud workload balancing).

    \item Programmed in C/C++ and Perl.

    \item Lectured at Saint Petersburg Computer Science Center (Information Storage and Management course) - \url{www.compscicenter.ru}, \\
	    lectures are made public at \url{www.lektorium.tv/course/?id=22928}.

    \end{itemize}

  \item{\bfseries Motorola ZAO, Saint Petersburg, Russia} \hfill February 20, 2006 - May 25, 2008 \\
    {\em Senior Software Developer, Project/Team Leader}  

    \begin{itemize}

    \item Designed and developed several components of Access Point for WiMAX network, programmed 
       DHCP, ICMP, ARP, Mobile IP, link managers. Wrote requirements and software architecture documents. 

    \item Leaded Access Point Test Environment development team - organized work, communicated with
       customers. 

    \item Prepared and held seminars "Mobile IPv6", "Secure Programming in C/C++", "Robust Header Compression".

    \item Participated in hiring activities, conducted technical interviews 
        (as expert for C/C++ and UNIX/Linux development).

    \item Mentored and consulted colleagues in areas related to UNIX/Linux development, C/C++ and networking. 

    \end{itemize}

  \item{\bfseries exorbyte GmbH, Konstanz, Germany} \hfill January 1, 2004 - December 31, 2005 \\ 
    {\em Senior Software Developer} for Approximate Search Technologies
    
    \begin{itemize}
      
    \item Developed, integrated and supported MatchBox and MatchMaker
      - {approximate} search and matching solutions for structured data providing
      exceptional matching quality and ultra-fast approximate multi-field search.
    
    \item Added UTF-8 support to MatchBox and MatchMaker (search through multilingual fields,
      sentences, character mappings and expansions).

    \item Designed and developed MatchMaker's license server - allowing for secure communication
      between system components, different licensing modes (performance share and accounting),
      mobility (host independent licensing), intellectual property protection (based on hardware keys).

    \item Programmed hardware keys (from WIBU SYSTEMS), designed and developed various software metering
      and protection solutions.

    \item Ported MatchBox and MatchMaker suite to Linux and Sun Solaris Platforms, was responsible 
      for complete development process on UNIX platforms (design, coding, testing, builds, integration,
      consulting).

    \item Designed and developed project build system in TCL with support of hierarchical
      dependencies check and compilation of Java and C++ projects. 

    \item Configured and administered Intranet, Linux and Solaris workstations, NFS, YP(NIS), 
      DNS, other system and network services.

    \item Programmed in C, C++, TCL/TK and Bash.

    \end{itemize}

  \item{\bfseries SchlumbergerSema, Germany} \hfill March 1, 2001 - December 31, 2003 \\ 
    (in past Sema-Telecoms, LHS, now - Ericsson, Frankfurt am Main), \\
    {\em Software Engineer} at Release Support Center for Billing Systems Solutions.
    
    \begin{itemize}
      
    \item Second and third level support of Business Support and Control System (BSCS),
      from version 5.10 to 7.00.
      
    \item Debugged, tuned and customized billing kernel of BSCS.

    \item Solved many severe problems of GSM operators, fast and reliably.

    \item Ported BSCS processing chain to HP Itanium 2 platform.
      
    \item Worked with many UNIX operating systems, compilers, debuggers and tools.

    \item Programmed in C and C++, Oracle Embedded SQL.

    \end{itemize}
    
  \item{\bfseries University of Karlsruhe, Germany} \hfill October 1, 1998 - February 28, 2001 \\ 
    {\em Researcher} at Institute for Process control and Robotics.
    
    \begin{itemize}
      
    \item Simulated work of an industrial robot with help of CAD/CAM system ``RobCAD''.

    \item Modeled workcells and connected simulation software with the control system of the robot.

    \item Worked closely with SGI IRIX and Linux Systems for the design and implementation 
      of a multi-agent control system for robot cells (in C++). 
      Used CORBA for interconnection of components. This work is fully described
      in ``Rembold, Derk: Kommisioniersystem mit automatischer Zuordnung von Greifwerkzeugen
      f\"{u}r die flexible Handhabung von Objekten. CGA-Verlag, 2001. ISBN 3-89863-028-5''.

    \item Programmed user interfaces with TCL/TK.

    \item Visualized the robot cell and programmed other applications with OpenGL.

    \item Integrated laser scanner into the robot cell.

    \item Worked on strategies of manipulation of unknown objects with different types
      of grippers.

    \item Was involved into the research project DIAMOND (Distributed Architecture for 
      Monitoring and Diagnosis) that was
      founded by European Commission. Project was related with diagnostics of robots.

    \item Implemented part of a distributed multi-agent architecture in Java (database 
      interface for DIAMOND).

    \end{itemize}

  \end{description}

  \vspace{1ex}

  \uline{ \bfseries{EDUCATION} }

  \begin{description}

  \item{ September 1, 2011 - October 23, 2013 \bfseries \\ 
	  Cork Institute of Technology, Ireland}\\
  Master of Science in Cloud Computing, First Class Honours. Attended courses - Software Engineering, Scripting (VMware vSphere Perl SDK), Management of Virtual Environments (vSphere), Cloud Strategy Planning and Management, Computing Research and Practice, Data Centre Networking, Cloud Security, Cloud Storage Infrastructures. Diploma thesis - "Tenant Behavior-driven Scheduler in OpenStack Cloud".

  \item{ April 30, 2007 - October 23, 2007 \bfseries \\
    Project Management Institute, Saint Petersburg Chapter, Russia} \\
  Postgraduate studies of Project Management (ANSI PMBOK). Attended courses -
  Project Management (Scope, Integration, Schedule, Quality, Communication, Procurement,
  Human Resources, Cost, Risks), Quality Management (Planning, Assurance, Control). 

  \item{ October 1, 1998 - February 28, 2001 \bfseries \\
    Institute for Process Control and Robotics, University of Karlsruhe, Germany} \\
  Exchange student and researcher. Research and development in area of industrial robotics.

  \item{ September 1, 1993 - October 15, 1999 \bfseries \\
    Saint Petersburg State University of Aerospace Instrumentation, Russia} \\
  Diploma Engineer in Robotics (Master), Control Systems for Robots and Complex Robot Cells. 
  Key courses: electrical engineering, applied mechanics, design and modeling of robot 
  systems and microprocessor control systems.

  \end{description}

  \vspace{1ex}

  \uline{ \bfseries{ATTENDED COURSES} }

  \begin{itemize}
  
  \item Java Programming, O'Reilly School of Technology.
 
  \item Many EMC and storage technology related trainings (Celerra, Clariion, Virtualization, Information Storage and Management).

  \item Motorola provided courses - Secure Programming, Productivity Measurement Systems,
        Record Management, Behavioral Interviewing, Unified Modeling Language, 
        Six Sigma Foundations, Project Management Workshop for Project Leaders.
 
  \item Oracle SQL Optimization, by Oracle.
    
  \item C++ Advanced Course, by K\"{o}lsch \& Altman, Software \& Management Consulting GmbH.
   
  \item Rational Purify for Unix, by Rational.

  \end{itemize}

  \vspace{1ex}

  \uline{ \bfseries{PUBLICATIONS} }

  \begin{itemize}
    \item A. O'Driscoll, V. Belogrudov, J. Carroll, K. Kropp, P. Walsh, P. Ghazal, R. Sleator.
      HBLAST: Parallelised sequence similarity - A Hadoop MapReducable basic local alignment search tool. In \textsl{Journal of Biomedical Informatics}, April 2015

    \item D. Rembold, V. Belogroudov, T. L\"{a}ngle, and H. W\"{o}rn. 
      Automatic selection of grippers for object handling. In \textsl{AMS 2000}, 
      Karlsruhe, Germany, November 2000.
      
    \item D. Rembold, V. Belogroudov, and H. W\"{o}rn. Object turning for bar code
      search. In \textsl{Proceedings of the IEEE/RSJ International Conference on
      Intelligent Robots and Systems}, Kagawa University, Takasamatsu, Japan,
      November 2000.

  \end{itemize}

  \vspace{1ex}

  \uline{ \bfseries {LANGUAGES} }

  \vspace{2ex}
  English - fluent, German - can read and speak, Russian - native
    

\end{document}
